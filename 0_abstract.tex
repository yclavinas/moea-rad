MOEA/D decomposes multi-objective problems into single-objective subproblems and solve them in parallel. In standard MOEA/D, all subproblems receive the same computational effort. However, as each subproblem is related to a different area of the objective space, it is expected that some subproblems are more difficult than others. Using Resource Allocation, MOEA/D could spend less effort on easier subproblems and more on harder ones, improving efficiency. In this paper, we address Resource Allocation that uses priority functions. They determine which subproblems should receive more computation resources. We propose the MOEA/D-RAD, a MOEA/D that considers diversity in the decision space as the measure of priority among candidate solutions. We compare MOEA/D-RAD, MOEA/D-DE and MOEA/D-DRA on the bbob-biobj benchmark, composed of 55 functions grouped into 15 groups, based on the function properties. We investigate the performance of these three methods based on hypervolume and proportion of non-dominated solutions in all of these 15 groups. Exploratory experiments show that MOEA/D-RAD obtained the best hypervolume in 24 functions. In particular MOEA/D-RAD obtained a good performance in groups characterized by moderated and weakly-structured groups.These results validate the effectiveness of using diversity in the objective space as priority function in the MOEA/D framework.