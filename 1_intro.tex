\section{Introduction}

Multi-objective Optimization Problems (MOP) are maximization (or minimization) problems characterized by multiple, conflicting objective functions. It arises in real world applications that require a compromise among multiple objectives. An MOP can be summarized as


%These $m$ multiple objective functions must be optimized simultaneously:

\vspace{-1em}
\begin{align}\label{min_problem}
\text{minimize} f(x) = (f_1(x), ..., f_{m}(x)), \text{$x \in \mathbb{R}^{D}$},
\end{align}

where $m$ is the number of objective functions and $\mathbb{R}^m$ is the objective function space. $x \in \mathbb{R}^{D} = \{x_1, x_2, ..., x_D\}$ is a D-dimensional vector which represents a candidate solution with ${D}$ variables, $f: \mathbb{R}^{D} \rightarrow \mathbb{R}^{m}$ is a vector of objective functions.

These objectives often conflict with each other, as there is usually no solution in $\mathbb{R}^{D}$ that minimizes all the objectives at the same time. Consequently, the goal of the MOP optimization algorithm is to find the approximate set of solutions that balance the different objectives in an optimal way.

This balance is defined by the concept of ``pareto dominance''. Given two solutions vectors $u, v$ in $\mathbb{R}^{D}$, we said that $u$  Pareto-dominates $v$, denoted by $f(u) \prec f(v)$, if and only if $f_k(u) \leq f_k(v), \forall_k \in \{1,..., m\}$ and $ f(u) \neq f(v)$. Likewise, a solution $x \in \mathbb{R}^{D}$ is considered Pareto-Optimal if there exists no other solution $y \in \mathbb{R}^{D}$ such that $f(y) \succ f(x)$, i.e., if $x$ is non-dominated in the feasible decision space. A non-dominated solution exists if no other solution provides a better trade-off in all objectives. Consequently, the set of all Pareto-Optimal solutions is known as the Pareto-Optimal Set (PS), while the image of this set is referred to as the Pareto-optimal Front (PF).% We recall that weak dominance ($A \succeq B$) means that any solution in set B is weakly dominated by a solution in set A. However, this does not rule out equality, because $A \succeq A$ for all approximation sets $A$~\cite{zitzler2003performance}.
%\\
%\vspace{-1em}
\begin{equation}
PS = \{x \in \mathbb{R}^{D} | \nexists y \in \mathbb{R}^{D} : f(y) \succ f(x)  \},
\end{equation}
%\vspace{-1em}
\begin{equation}
PF = \{f(x) | x \in PS \}.
\end{equation}

% Mention that MOEAD is a good approach for MOP
The Multi-Objective Evolutionary Algorithm Based on Decomposition
(MOEA/D)~\cite{zhang2007moea} is an effective algorithm for solving MOPs. The
key idea of the MOEA/D is that the multi-objective optimization problem is
decomposed into a set of single objectives subproblems All subproblems
are then solved in parallel.

% Mention that MOEAD all problems are treated equally, but in actually they are
In the original MOEA/D, all subproblems are treated uniformly, in the sense that
all of them receive the same computational effort. However, it has been observed
that some subproblems are harder than others, and take more effort to converge
to an optimal solution~\cite{zhou2016all}. Because of this, \emph{Resource
	Allocation} approaches have been proposed to allocate different amount of
computational effort to different subproblems, based on an estimation of the
relative difficulty of each subproblem~\cite{zhou2016all},~\cite{zhang2009performance},	~\cite{kang2018collaborative}. The most popular MOEA/D algorithm related to Resource Allocation is the MOEA/D-DRA~\cite{zhang2009performance}. It uses the Relative Improvement for estimating subproblem
difficulty, which calculates how much a subproblem
has improved in recent iterations.


% Motivation for this approach
Here, we propose a new approach for estimating difficulty and
calculating priority in Resource Allocation for MOEA/D. Our approach uses the
idea of \emph{diversity} in objective space to calculate
the priority of solutions. Our motivation for this choice is that the quality of
a MOP solution set is often evaluated by the diversity in the objective space.
If we assign higher priority for regions with lower diversity, we are
encouraging the algorithm to spend more computational effort in regions that are
not yet well explored.

In this paper, we define a priority function based on diversity on the objective space using the MRDL, proposed by Gee~\cite{gee2015online}. The MRDL is an online diversity metric based on a geometrical perspective and indicates the loss of diversity related to a solution to the whole population. We understand that this priority function is able to monitor diversity during the execution of the algorithm guiding the search behavior of the algorithm.


We compare the new approach with the MOEA/D-DRA and with the standard MOEA/D (with no Resource Allocation). The results show that a priority function focused on decision space lead to better results on the metric Hypervolume (HV), for some groups of problems.


% The diversity on the decision space shows better performance on the benchmark function,  generally being better than using the Relative Improvement. To our surprise, priority function on objective space did not perform so well, barely improving the results compared with no Resource Allocation at all. On the Lunar Landing problem, all methods achieved similar HV values but, the proposed priority function obtained higher proportion of feasible solutions.
