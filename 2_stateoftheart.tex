\section{Related Works}

\subsection{Priority functions}

%Following the discussion opened in the introduction, we need to define an utility function to allocate the number of evaluations to subproblems. 
We define priority functions (also called utility functions) as one way of establishing preferences between solutions for Resource Allocation~\cite{chankong1983multiobjective}. These functions are used to decide how to allocate computational resources among subproblems by monitoring the algorithm search and guiding the distribution over iterations~\cite{cai2015external}. 

Only a few studies have been concerned with Resource Allocation in MOEA/D. We highlight two groups. The first is composed by MOEA/D-GRA~\cite{zhou2016all}, MOEA/D-DRA~\cite{zhang2009performance} and in the Two-Level Stable Matching-Based Selection in MOEA/D~\cite{nasir2011improved}. The other is composed by EAG-MOEA/D~\cite{cai2015external} and MOEA/D-CRA~\cite{kang2018collaborative}.

According to Zhou and Zhang~\cite{zhou2016all}, MOEA/D-GRA could be seen as an extension of MOEA/D-DRA and MOEA/D-AMS~\cite{chiang2011moea}. They reason that all these algorithms use a very similar priority function and that MOEA/D-GRA can simulate the behavior of MOEA/D-DRA or MOEA/D-AMS by changing the values of a single parameter. 

This priority function is named as the Relative Improvement (R.I.) and defines the priority values of each subproblem $i=1,...,N$, as

\begin{equation}\label{priority}
\delta_i = \dfrac{\text{old function value}-\text{new function value}}{\text{old function value}}.
\end{equation}

Where new function value is the value at the current iteration ($T$) and old function value is the value at iteration ($T - \Delta T$).

For MOEA/D-GRA $u_i = \delta_i$, but in MOEA/D-DRA, as well as in the Two-Level Stable Matching-Based Selection in MOEA/D, a second equation is used,

\[
u_i= 
\begin{cases}

(0.95 + 0.05 \cdot \frac{\delta_i}{0.001} \cdot u_i), & \text{if } \delta_i > 0.001,
\\
1,              & \text{otherwise}

\end{cases}\label{priority2}
\]
The R.I. is based the assumption that if a subproblem has been improved over the last $\Delta T$ iteration (\textit{old function value}), it should have a high probability of being improved over the next iterations. 

The priority function used EAG-MOEA/D~\cite{cai2015external} and MOEA/D-CRA~\cite{kang2018collaborative} differ from the ones in the MOEA/D-GRA group. In their case, the framework keeps two populations: one working population, and one external archive. This priority function estimates priorities for a subproblem given the number of solutions from that subproblem that are in the external archive.

Together these studies indicate that it is worth monitoring the algorithm behavior and guiding its search, but it is unclear how the choice of priority functions influence the results. In all Resource Allocation works mentioned above, the choice of priority function was just one of multiple changes applied to the base framework. For example, in Zhang et al. used a 10-tournament selection in MOEA/D-DRA~\cite{zhang2009performance}, while Zhou and Zhang used a new replacement strategy in MOEA/D-GRA~\cite{zhou2016all}. Chiang in MOEA/D-AMS proposes an adaptive mating selection mechanism as it dynamically adjusts the mating pools of individuals~\cite{chiang2011moea}. Finally, both the studies of Cai and Lai in  EAG-MOEA/D~\cite{cai2015external} and Kang et al. in MOEA/D-CRA~\cite{kang2018collaborative} used an archive population.

We focus on improving the performance of the standard MOEA/D by using priority functions, given the recent success of addressing Resource Allocation with priority functions. We believe in the idea that diversity is a critical issue in the search process for any multi-objective algorithm. Therefore, we consider using priority functions to address lack of diversity aiming to make solutions better spread among each other. The proposed priority function focus on how to better spread solutions the Pareto Set (diversity on the decision space). For that, we define the MRDL priority function.



\subsection{Diversity Metric}


One way to measure diversity is to use metrics that evaluate MOPs solvers. The hypervolume indicator (HV)~\cite{zitzler1998multiobjective} and the Inverted Generational Distance (IGD)~\cite{zhang2008rm} are frequently used as metrics to evaluate such solvers. However they include information about both quality of the solutions and diversity in a single metric.

Among the metrics that only measure diversity, there are mainly two groups. The offline group, that calculate the diversity after the execution of the algorithm, while online group, that calculate the diversity during the execution of the algorithm. We are interested in measuring diversity during the execution of the algorithm, therefore we  briefly introduce some studies that are part of the online group.
%
% The offline group is composed of: Chi-square-like deviation~\cite{deb1989genetic}; Spacing method~\cite{scott1995fault}; Uniformly distribution index~\cite{tan2002evolutionary}; Entropy approach~\cite{farhang2002diversity}; Grid diversity metric~\cite{deb2002running}; sparsity measure~\cite{deb2003fast}. These published studies need knowledge of the PF or the ideal vector. While the online group is composed of: sigma method~\cite{mostaghim2003strategies}  (PF lies in the positive objective space); entropy of the solutions by using Parzen window density estimation-\cite{tan2008evolutionary} (sensitive to kernel width); maximum relative diversity loss~\cite{gee2015online} (expensive $O(N^2)$, with $N$ being the size of the parent population).

The online group includes: sigma method~\cite{mostaghim2003strategies}  (PF lies in the positive objective space); entropy of the solutions by using Parzen window density estimation-\cite{tan2008evolutionary} (sensitive to kernel width); and maximum relative diversity loss, MRDL, ~\cite{gee2015online} (expensive $O(N^2)$, with $N$ being the size of the parent population).



In this work we chose to apply the MRDL as the strategy to measure diversity on the objective space. This is an online diversity metric estimates the diversity loss of a solution to the whole population~\cite{gee2015online}. High values indicate the existence of similar solutions or that the offspring solution is close to the convergence direction. The further an objective vector of a solution is from the convergence direction, the more it contributes for the diversity of the approximated the Pareto Front. The MRDL is the maximum value for Relative Diversity Loss (RDL) of each solution.



