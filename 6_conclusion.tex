\section{Conclusion}

In this research, we propose a new algorithm for Resource Allocation in
Multi-Objective Optimization, MOEA/D-RAD. The performance of this algorithm is
studied on the bbob-biobj benchmark function set. This algorithm differs from
standard MOEA/D by the use of Resource Allocation techniques, which allocate
computational effort proportional to each subproblem's difficulty.

MOEA/D-RAD performs Resource Allocation using MRDL as a priority function based
on a geometrical perspective. In this sense, MOEA/D-RAD allocates computational
resources to different subproblems based on its contribution to the overall
diversity of the population.

We compared this new approach with MOEA/D-DRA, which is another
Resource Allocation strategy. MOEA/D-DRA allocates computational resources
based on the difference between parent and child solutions. We also compared
with MOEA/D-DE without Resource Allocation strategy.

The experimental results showed that MOEA/D-RAD performed well on several  of
the test problems. In particular, we observed that MOEA/D-RAD performed well in
functions that are moderate-conditioned and weakly-structured.  On the other
hand, MOEA/D-DRA showed the highest rate of non-dominated  solutions in the
final solution set.

In comparison with MOEA/D-DRA, the Resource Allocation by the proposed
MOEA/D-RAD was less extreme. This might explain why MOEA/D-RAD and MOEA/D-DE
were usually successful in similar functions, although MOEA/D-RAD always  had a
better proportion of non-dominated solutions, and often achieved better
hypervolume values than MOEA/D-DE. We understand that this shows that using the
MRDL function as the priority function achieved its goal of promoting diversity
in the solution set.

Overall, the findings of this work reinforce the need of Resource Allocation for
MOEA/D, and showcase particular types of functions where Resource Allocation
techniques would be most useful. The differences between MOEA/D-DRA and
MOEA/D-RAD also  suggests that the choice of priority function is a critical
component of Resource Allocation and that this choice needs to be informed by
the characteristics of the problem being solved.

Our results indicate that the MRDL priority function might be a reasonable
choice in problems where at least one objective is moderate-conditioned or
weakly structured. However, as this observation came from a post-hoc analysis,
further exploration is crucial to confirm this finding.

Other future works include further consideration into priority functions  (such
as objective-space and decision-space priority functions) as well as  Resource
Allocation and priority functions based on Archive techniques,  such as
MOEA/D-CRA~\cite{kang2018collaborative}.
